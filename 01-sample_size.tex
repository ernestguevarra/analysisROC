\documentclass[12pt,a4paper]{article}
\usepackage{lmodern}
\usepackage{amssymb,amsmath}
\usepackage{ifxetex,ifluatex}
\usepackage{fixltx2e} % provides \textsubscript
\ifnum 0\ifxetex 1\fi\ifluatex 1\fi=0 % if pdftex
  \usepackage[T1]{fontenc}
  \usepackage[utf8]{inputenc}
\else % if luatex or xelatex
  \ifxetex
    \usepackage{mathspec}
  \else
    \usepackage{fontspec}
  \fi
  \defaultfontfeatures{Ligatures=TeX,Scale=MatchLowercase}
\fi
% use upquote if available, for straight quotes in verbatim environments
\IfFileExists{upquote.sty}{\usepackage{upquote}}{}
% use microtype if available
\IfFileExists{microtype.sty}{%
\usepackage{microtype}
\UseMicrotypeSet[protrusion]{basicmath} % disable protrusion for tt fonts
}{}
\usepackage[margin=2cm]{geometry}
\usepackage{hyperref}
\PassOptionsToPackage{usenames,dvipsnames}{color} % color is loaded by hyperref
\hypersetup{unicode=true,
            pdftitle={Notes on sample sizes required for Receiver Operator Characteristic (ROC) curves},
            colorlinks=true,
            linkcolor=blue,
            citecolor=blue,
            urlcolor=blue,
            breaklinks=true}
\urlstyle{same}  % don't use monospace font for urls
\usepackage{natbib}
\bibliographystyle{plainnat}
\usepackage{longtable,booktabs}
\usepackage{graphicx,grffile}
\makeatletter
\def\maxwidth{\ifdim\Gin@nat@width>\linewidth\linewidth\else\Gin@nat@width\fi}
\def\maxheight{\ifdim\Gin@nat@height>\textheight\textheight\else\Gin@nat@height\fi}
\makeatother
% Scale images if necessary, so that they will not overflow the page
% margins by default, and it is still possible to overwrite the defaults
% using explicit options in \includegraphics[width, height, ...]{}
\setkeys{Gin}{width=\maxwidth,height=\maxheight,keepaspectratio}
\IfFileExists{parskip.sty}{%
\usepackage{parskip}
}{% else
\setlength{\parindent}{0pt}
\setlength{\parskip}{6pt plus 2pt minus 1pt}
}
\setlength{\emergencystretch}{3em}  % prevent overfull lines
\providecommand{\tightlist}{%
  \setlength{\itemsep}{0pt}\setlength{\parskip}{0pt}}
\setcounter{secnumdepth}{5}
% Redefines (sub)paragraphs to behave more like sections
\ifx\paragraph\undefined\else
\let\oldparagraph\paragraph
\renewcommand{\paragraph}[1]{\oldparagraph{#1}\mbox{}}
\fi
\ifx\subparagraph\undefined\else
\let\oldsubparagraph\subparagraph
\renewcommand{\subparagraph}[1]{\oldsubparagraph{#1}\mbox{}}
\fi

%%% Use protect on footnotes to avoid problems with footnotes in titles
\let\rmarkdownfootnote\footnote%
\def\footnote{\protect\rmarkdownfootnote}

%%% Change title format to be more compact
\usepackage{titling}

% Create subtitle command for use in maketitle
\newcommand{\subtitle}[1]{
  \posttitle{
    \begin{center}\large#1\end{center}
    }
}

\setlength{\droptitle}{-2em}

  \title{Notes on sample sizes required for Receiver Operator Characteristic
(ROC) curves}
    \pretitle{\vspace{\droptitle}\centering\huge}
  \posttitle{\par}
    \author{}
    \preauthor{}\postauthor{}
      \predate{\centering\large\emph}
  \postdate{\par}
    \date{13 November 2018}

\usepackage{booktabs}
\usepackage[table]{xcolor}
\usepackage{color}
\usepackage{tcolorbox}
\usepackage{float}
\usepackage{setspace}
\usepackage{longtable}
%\usepackage{amsmath}
%\usepackage{mathtools}

\onehalfspacing

%\raggedbottom

\graphicspath{ {icons/} }

\newenvironment{rmdremind}
  {\begin{tcolorbox}[width=\textwidth, 
                     colback = {white}, 
                     title = {\textbf{Remember}}, 
                     colbacktitle = lightgray,
                     coltitle = black]
  \begin{includegraphics}[scale = 1]{remind.png}
  \begin{itemize}}
  {\end{itemize}
  \end{includegraphics}
  \end{tcolorbox}}

\newenvironment{rmdnote}
  {\begin{tcolorbox}[width=\textwidth, 
                     colback = {white}, 
                     title = {\textbf{Note}}, 
                     colbacktitle = lightgray,
                     coltitle = black]
  \begin{includegraphics}[scale = 1]{pencil.png}}
  {\end{includegraphics}
  \end{tcolorbox}}
  
\newenvironment{rmdcalc}
  {\begin{tcolorbox}[width=\textwidth, 
                     colback = {white}, 
                     title = {\textbf{Calculations}}, 
                     colbacktitle = lightgray,
                     coltitle = black]
  \begin{includegraphics}[scale = 2]{pencil.png}}
  {\end{includegraphics}
  \end{tcolorbox}}
  
\newenvironment{rmdexercise}
  {\begin{tcolorbox}[width=\textwidth, 
                     colback = {white}, 
                     title = {\textbf{Exercise}}, 
                     colbacktitle = lightgray,
                     coltitle = black]
  \begin{includegraphics}[scale = 1]{exercise.png}}
  {\end{includegraphics}
  \end{tcolorbox}}
  
\newenvironment{rmdbox}
  {\begin{tcolorbox}[width=\textwidth, 
                     colback = {white}, 
                     title = {\textbf{Exercise}}, 
                     colbacktitle = lightgray,
                     coltitle = black]
  \begin{includegraphics}[scale = 1]{pencil.png}}
  {\end{includegraphics}
  \end{tcolorbox}}
  
\newenvironment{rmdinfo}
  {\begin{tcolorbox}[width=\textwidth, 
                     colback = {white}, 
                     title = {\textbf{Info}}, 
                     colbacktitle = lightgray,
                     coltitle = black]
  \begin{includegraphics}[scale = 1]{info.png}}
  {\end{includegraphics}
  \end{tcolorbox}}  
  
\newenvironment{rmdwarning}
  {\begin{tcolorbox}[width=\textwidth, 
                     colback = {white}, 
                     title = {\textbf{Warning}}, 
                     colbacktitle = lightgray,
                     coltitle = black]
  \begin{includegraphics}[scale = 1]{warning.png}}
  {\end{includegraphics}
  \end{tcolorbox}}

\newenvironment{rmdcaution}
  {\begin{tcolorbox}[width=\textwidth, 
                     colback = {white}, 
                     title = {\textbf{Caution}}, 
                     colbacktitle = lightgray,
                     coltitle = black]
  \begin{includegraphics}[scale = 1]{warning.png}}
  {\end{includegraphics}
  \end{tcolorbox}}

\newenvironment{rmddownload}
  {\begin{tcolorbox}[width=\textwidth, 
                     colback = {white}, 
                     title = {\textbf{Download}}, 
                     colbacktitle = lightgray,
                     coltitle = black]
  \begin{includegraphics}[scale = 1]{download.png}}
  {\end{includegraphics}
  \end{tcolorbox}}

\usepackage{amsthm}
\newtheorem{theorem}{Theorem}[section]
\newtheorem{lemma}{Lemma}[section]
\theoremstyle{definition}
\newtheorem{definition}{Definition}[section]
\newtheorem{corollary}{Corollary}[section]
\newtheorem{proposition}{Proposition}[section]
\theoremstyle{definition}
\newtheorem{example}{Example}[section]
\theoremstyle{definition}
\newtheorem{exercise}{Exercise}[section]
\theoremstyle{remark}
\newtheorem*{remark}{Remark}
\newtheorem*{solution}{Solution}
\begin{document}
\maketitle

\hypertarget{calculating-sample-size-needed-to-estimate-area-under-an-roc-curve-given-a-specified-ci-and-power}{%
\section{Calculating sample size needed to estimate area under an ROC
curve given a specified CI and
power}\label{calculating-sample-size-needed-to-estimate-area-under-an-roc-curve-given-a-specified-ci-and-power}}

\hypertarget{calculating-ratio-of-number-of-non-cases-to-cases}{%
\subsection{Calculating ratio of number of non-cases to
cases}\label{calculating-ratio-of-number-of-non-cases-to-cases}}

We first need to calculate \(k\) which is the ratio of number of
non-cases to cases. This can be calculated as follows:

\[ \begin{aligned}
k ~ & = ~ \frac{1 ~ - ~ PREV_p}{PREV_p} \\
\\
where: & \\
\\
PREV_p ~ & = ~ \text{prevalence of the cases in the population}
\end{aligned} \]

Given that we don't have specific values for the prevalence of children
demonstrating appropriate infant and young child feeding, we make an
educated guess of what this prevalence can be. It is likely that the
practice of appropriate infand and young child feeding is low and is
probably at best at the 20\% level.

Using this, we estimate \(k\) as follows:

\[ \begin{aligned} 
k ~ & = ~ \frac{1 ~ - ~ PREV_p}{PREV_p} \\
\\
& = ~ \frac{1 ~ - ~ 0.2}{0.2} \\
\\
& = ~ 4
\end{aligned} \]

\hypertarget{calculating-the-binomial-distribution-parameter}{%
\subsection{Calculating the binomial distribution
parameter}\label{calculating-the-binomial-distribution-parameter}}

We then need to calculate the binomial distribution parameter \(A\) as
follows:

\[ \begin{aligned}
A ~ & = ~ \phi ^ {-1}(AUC) ~ \times ~ 1.414 \\
\\
where: & \\
\\
\phi ^ {-1} ~ & = ~ \text{inverse of the cumulative normal disribution function} \\
\\
AUC ~ & = ~ \text{expected area under the curve}
\end{aligned} \]

Since we do not have prior knowledge of the AUC from previous studies,
we set AUC at 0.80 which is the AUC value we would assume that would
show agreement between ICFI and indicators for nutrition and energy
adequacy. This gives us the following \(A\) parameter:

\[ \begin{aligned}
A ~ & = ~ \phi ^ {-1}(AUC) ~ \times ~ 1.414 \\
\\
& = ~ \phi ^ {-1}(0.80) ~ \times ~ 1.414 \\
\\
& = ~ 0.25 ~ \times ~ 1.414 \\
\\
& = ~ 1.190052
\end{aligned} \]

\hypertarget{calculating-the-variance-functiion}{%
\subsection{Calculating the variance
functiion}\label{calculating-the-variance-functiion}}

To be able to continue with the calculations, we will need to calculate
variance function (\(VF\)). The variance function (\(VF\)) is calculated
as follows:

\[ \begin{aligned}
VF ~ & = ~ 0.0099 ~ \times ~ e ^ {-A ~ \times ~ A/2} ~ \times ~ \Bigg [ ~ (5 ~ \times ~ A ^ 2 ~ + ~ 8) ~ + ~ \frac{(A ^ 2 ~ + ~ 8)}{k} ~ \Bigg ] \\
\\
where: & \\
\\
A ~ & = ~ 1.190052 \\
\\
k ~ & = ~ 4
\end{aligned}\]

Using the values calculated for \(A\) and \(k\) previously, we calculate
\(VF\) as follows:

\[ \begin{aligned}
VF ~ & = ~ 0.0099 ~ \times ~ e ^ {-A ~ \times ~ A/2} ~ \times ~ \Bigg [ ~ (5 ~ \times ~ A ^ 2 ~ + ~ 8) ~ + ~ \frac{(A ^ 2 ~ + ~ 8)}{k} ~ \Bigg ] \\
\\
& = ~ 0.0099 ~ \times ~ 0.4925731 ~ \times ~ \Bigg [ ~ (5 ~ \times ~ 1.416225 ~ + ~ 8) ~ + ~ \frac{(1.416225 ~ + ~ 8)}{4} ~ \Bigg ] \\
\\
& = ~ 0.004876474 ~ \times ~ [ ~ 15.08112 ~ + ~ 2.354056 ~] \\
\\
& = ~ 0.004876474 ~ \times 17.43518 \\
\\
& = ~ 0.0850222
\end{aligned} \]

\hypertarget{calculating-number-of-cases-needed-in-the-study-sample}{%
\subsection{Calculating number of cases needed in the study
sample}\label{calculating-number-of-cases-needed-in-the-study-sample}}

Now we can calculate the number of cases (i.e.~those who have the
condition of interest). For the case of ICFI or IYCF, this will be those
children who exhibit or demonstrate appropriate infant and young child
feeding. This can be calculated as follows:

\[ \begin{aligned}
N ~ & = ~ \frac{Z_{\alpha/2} ^ 2 ~ \times ~ VF}{L ^ 2} \\
\\
where: & \\
\\
Z_{\frac{\alpha}{2}} ~ & = ~ 1.96 ~ \text{for a 95\% CI} \\
\\
L ~ & = 0.05 ~ \text{(desired half-width of the CI)} \\
\\
VF ~ & = ~ 0.0850222  \\
\end{aligned} \]

Using the value for \(VF\) calculated previously, we calculate \(N\) as
follows:

\[ \begin{aligned}
N ~ & = ~ \frac{Z_{\alpha/2} ^ 2 ~ \times ~ VF}{L ^ 2} \\
\\
& = ~ \frac{1.96 ^ 2 ~ \times ~ 0.0850222}{0.05^ 2} \\
\\
& = ~ 130.6485 ~ \approx ~ 131
\end{aligned} \]

\hypertarget{total-sample-size}{%
\subsection{Total sample size}\label{total-sample-size}}

The total sample size \(n\) needed for the ROC analysis can then be
calculated as follows:

\[ \begin{aligned}
n ~ & = ~ N ~ \times ~ (1 ~ + ~ k)\\
\\
where: & \\
\\
N ~ & = ~ 131 \\
\\
k ~ & = ~ 4
\end{aligned} \]

Using the \(N\) and \(k\) values calculated earlier, \(n\) is calculated
as follows:

\[ \begin{aligned}
n ~ & = ~ N ~ \times ~ (1 ~ + ~ k)\\
\\
& = ~ 131 ~ \times ~ 5 \\
\\
& = ~ 655
\end{aligned} \]

We will need a sample size of about \textbf{655} children about
\textbf{131} of which will be children who are practising appropriate
infant and young child feeding.

\bibliography{bibliography.bib}


\end{document}
