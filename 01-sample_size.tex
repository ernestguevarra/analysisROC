\documentclass[12pt,a4paper]{article}
\usepackage{lmodern}
\usepackage{amssymb,amsmath}
\usepackage{ifxetex,ifluatex}
\usepackage{fixltx2e} % provides \textsubscript
\ifnum 0\ifxetex 1\fi\ifluatex 1\fi=0 % if pdftex
  \usepackage[T1]{fontenc}
  \usepackage[utf8]{inputenc}
\else % if luatex or xelatex
  \ifxetex
    \usepackage{mathspec}
  \else
    \usepackage{fontspec}
  \fi
  \defaultfontfeatures{Ligatures=TeX,Scale=MatchLowercase}
\fi
% use upquote if available, for straight quotes in verbatim environments
\IfFileExists{upquote.sty}{\usepackage{upquote}}{}
% use microtype if available
\IfFileExists{microtype.sty}{%
\usepackage{microtype}
\UseMicrotypeSet[protrusion]{basicmath} % disable protrusion for tt fonts
}{}
\usepackage[margin=2cm]{geometry}
\usepackage{hyperref}
\PassOptionsToPackage{usenames,dvipsnames}{color} % color is loaded by hyperref
\hypersetup{unicode=true,
            pdftitle={Notes on sample sizes required for Receiver Operator Characteristic (ROC) curves},
            colorlinks=true,
            linkcolor=blue,
            citecolor=blue,
            urlcolor=blue,
            breaklinks=true}
\urlstyle{same}  % don't use monospace font for urls
\usepackage{natbib}
\bibliographystyle{plainnat}
\usepackage{longtable,booktabs}
\usepackage{graphicx,grffile}
\makeatletter
\def\maxwidth{\ifdim\Gin@nat@width>\linewidth\linewidth\else\Gin@nat@width\fi}
\def\maxheight{\ifdim\Gin@nat@height>\textheight\textheight\else\Gin@nat@height\fi}
\makeatother
% Scale images if necessary, so that they will not overflow the page
% margins by default, and it is still possible to overwrite the defaults
% using explicit options in \includegraphics[width, height, ...]{}
\setkeys{Gin}{width=\maxwidth,height=\maxheight,keepaspectratio}
\IfFileExists{parskip.sty}{%
\usepackage{parskip}
}{% else
\setlength{\parindent}{0pt}
\setlength{\parskip}{6pt plus 2pt minus 1pt}
}
\setlength{\emergencystretch}{3em}  % prevent overfull lines
\providecommand{\tightlist}{%
  \setlength{\itemsep}{0pt}\setlength{\parskip}{0pt}}
\setcounter{secnumdepth}{5}
% Redefines (sub)paragraphs to behave more like sections
\ifx\paragraph\undefined\else
\let\oldparagraph\paragraph
\renewcommand{\paragraph}[1]{\oldparagraph{#1}\mbox{}}
\fi
\ifx\subparagraph\undefined\else
\let\oldsubparagraph\subparagraph
\renewcommand{\subparagraph}[1]{\oldsubparagraph{#1}\mbox{}}
\fi

%%% Use protect on footnotes to avoid problems with footnotes in titles
\let\rmarkdownfootnote\footnote%
\def\footnote{\protect\rmarkdownfootnote}

%%% Change title format to be more compact
\usepackage{titling}

% Create subtitle command for use in maketitle
\newcommand{\subtitle}[1]{
  \posttitle{
    \begin{center}\large#1\end{center}
    }
}

\setlength{\droptitle}{-2em}

  \title{Notes on sample sizes required for Receiver Operator Characteristic
(ROC) curves}
    \pretitle{\vspace{\droptitle}\centering\huge}
  \posttitle{\par}
    \author{}
    \preauthor{}\postauthor{}
      \predate{\centering\large\emph}
  \postdate{\par}
    \date{14 November 2018}

\usepackage{booktabs}
\usepackage[table]{xcolor}
\usepackage{color}
\usepackage{tcolorbox}
\usepackage{float}
\usepackage{setspace}
\usepackage{longtable}
%\usepackage{amsmath}
%\usepackage{mathtools}

\onehalfspacing

%\raggedbottom

\graphicspath{ {icons/} }

\newenvironment{rmdremind}
  {\begin{tcolorbox}[width=\textwidth, 
                     colback = {white}, 
                     title = {\textbf{Remember}}, 
                     colbacktitle = lightgray,
                     coltitle = black]
  \begin{includegraphics}[scale = 1]{remind.png}
  \begin{itemize}}
  {\end{itemize}
  \end{includegraphics}
  \end{tcolorbox}}

\newenvironment{rmdnote}
  {\begin{tcolorbox}[width=\textwidth, 
                     colback = {white}, 
                     title = {\textbf{Note}}, 
                     colbacktitle = lightgray,
                     coltitle = black]
  \begin{includegraphics}[scale = 1]{pencil.png}}
  {\end{includegraphics}
  \end{tcolorbox}}
  
\newenvironment{rmdcalc}
  {\begin{tcolorbox}[width=\textwidth, 
                     colback = {white}, 
                     title = {\textbf{Calculations}}, 
                     colbacktitle = lightgray,
                     coltitle = black]
  \begin{includegraphics}[scale = 2]{pencil.png}}
  {\end{includegraphics}
  \end{tcolorbox}}
  
\newenvironment{rmdexercise}
  {\begin{tcolorbox}[width=\textwidth, 
                     colback = {white}, 
                     title = {\textbf{Exercise}}, 
                     colbacktitle = lightgray,
                     coltitle = black]
  \begin{includegraphics}[scale = 1]{exercise.png}}
  {\end{includegraphics}
  \end{tcolorbox}}
  
\newenvironment{rmdbox}
  {\begin{tcolorbox}[width=\textwidth, 
                     colback = {white}, 
                     title = {\textbf{Exercise}}, 
                     colbacktitle = lightgray,
                     coltitle = black]
  \begin{includegraphics}[scale = 1]{pencil.png}}
  {\end{includegraphics}
  \end{tcolorbox}}
  
\newenvironment{rmdinfo}
  {\begin{tcolorbox}[width=\textwidth, 
                     colback = {white}, 
                     title = {\textbf{Info}}, 
                     colbacktitle = lightgray,
                     coltitle = black]
  \begin{includegraphics}[scale = 1]{info.png}}
  {\end{includegraphics}
  \end{tcolorbox}}  
  
\newenvironment{rmdwarning}
  {\begin{tcolorbox}[width=\textwidth, 
                     colback = {white}, 
                     title = {\textbf{Warning}}, 
                     colbacktitle = lightgray,
                     coltitle = black]
  \begin{includegraphics}[scale = 1]{warning.png}}
  {\end{includegraphics}
  \end{tcolorbox}}

\newenvironment{rmdcaution}
  {\begin{tcolorbox}[width=\textwidth, 
                     colback = {white}, 
                     title = {\textbf{Caution}}, 
                     colbacktitle = lightgray,
                     coltitle = black]
  \begin{includegraphics}[scale = 1]{warning.png}}
  {\end{includegraphics}
  \end{tcolorbox}}

\newenvironment{rmddownload}
  {\begin{tcolorbox}[width=\textwidth, 
                     colback = {white}, 
                     title = {\textbf{Download}}, 
                     colbacktitle = lightgray,
                     coltitle = black]
  \begin{includegraphics}[scale = 1]{download.png}}
  {\end{includegraphics}
  \end{tcolorbox}}

\usepackage{amsthm}
\newtheorem{theorem}{Theorem}[section]
\newtheorem{lemma}{Lemma}[section]
\theoremstyle{definition}
\newtheorem{definition}{Definition}[section]
\newtheorem{corollary}{Corollary}[section]
\newtheorem{proposition}{Proposition}[section]
\theoremstyle{definition}
\newtheorem{example}{Example}[section]
\theoremstyle{definition}
\newtheorem{exercise}{Exercise}[section]
\theoremstyle{remark}
\newtheorem*{remark}{Remark}
\newtheorem*{solution}{Solution}
\begin{document}
\maketitle

\hypertarget{calculating-sample-size-needed-to-estimate-area-under-an-roc-curve-given-a-specified-ci-and-confidence-interval}{%
\section{Calculating sample size needed to estimate area under an ROC
curve given a specified CI and confidence
interval}\label{calculating-sample-size-needed-to-estimate-area-under-an-roc-curve-given-a-specified-ci-and-confidence-interval}}

Following are the steps to calculating sample size needed to estimate
area under an ROC curve given a specified CI and confidence interval.
These steps are based on \citet{Obuchowski1118}, \citet{Obuchowski2005}.

\hypertarget{calculating-ratio-of-number-of-non-cases-to-cases}{%
\subsection{Calculating ratio of number of non-cases to
cases}\label{calculating-ratio-of-number-of-non-cases-to-cases}}

We first need to calculate \(k\) which is the ratio of number of
non-cases to cases. This can be calculated as follows:

\[ \begin{aligned}
k ~ & = ~ \frac{1 ~ - ~ PREV_p}{PREV_p} \\
\\
where: & \\
\\
PREV_p ~ & = ~ \text{prevalence of the cases in the population}
\end{aligned} \]

Given that we don't have specific values for the prevalence of children
demonstrating appropriate infant and young child feeding, we make an
educated guess of what this prevalence can be. It is likely that the
practice of appropriate infant and young child feeding is low and is
probably at best at the 20\% level. Demographic and Health Survey (DHS)
for Burkina Faso in 2010 show that for the whole country in total,
minimum acceptable diet (MAD) was at 3.1\% and about 2.4\% for rural
areas. In 2016, SMART surveys conducted in parts of Burkina Faso have
shown in increase in this indicator to about 20\%.

Using this, we estimate \(k\) as follows:

\[ \begin{aligned} 
k ~ & = ~ \frac{1 ~ - ~ PREV_p}{PREV_p} \\
\\
& = ~ \frac{1 ~ - ~ 0.2}{0.2} \\
\\
& = ~ 4
\end{aligned} \]

\hypertarget{calculating-the-binormal-distribution-parameter}{%
\subsection{Calculating the binormal distribution
parameter}\label{calculating-the-binormal-distribution-parameter}}

We then need to calculate the binormal distribution parameter \(A\).
\(A\) is one of two (the other being \(B\)) parameters that describe the
ROC curve. However, these parameters are rarely empirically estimated as
they represent unobserved binormal variables. However, for sample size
calculations, parameter \(A\) can be estimated based on an expected area
under the curve (AUC) of the ROC curve. The AUC corresponds to the
expected level of agreement or accuracy of the metric being tested with
the actual status or condition observed.

For our purposes, the AUC will be for how much agreement or accuracy
there is for MAD and/or good ICFI to predict mean nutrient and mean
energy adequacy. \# Given AUC, \(A\) can be calculated as follows:

\[ \begin{aligned}
A ~ & = ~ \phi ^ {-1}(AUC) ~ \times ~ 1.414 \\
\\
where: & \\
\\
\phi ^ {-1} ~ & = ~ \text{inverse of the cumulative normal disribution function} \\
\\
AUC ~ & = ~ \text{expected area under the curve}
\end{aligned} \]

Since we do not have prior knowledge of the AUC from previous studies,
we set AUC at 0.90 which is the AUC value we would assume that would
show best agreement between ICFI and indicators for nutrition and energy
adequacy. This gives us the following \(A\) parameter:

\[ \begin{aligned}
A ~ & = ~ \phi ^ {-1}(AUC) ~ \times ~ 1.414 \\
\\
& = ~ \phi ^ {-1}(0.90) ~ \times ~ 1.414 \\
\\
& = ~ 1.281552 ~ \times ~ 1.414 \\
\\
& = ~ 1.812114
\end{aligned} \]

\hypertarget{calculating-the-variance-function}{%
\subsection{Calculating the variance
function}\label{calculating-the-variance-function}}

To be able to continue with the calculations, we will need to calculate
the variance function (\(VF\)) of the accuracy estimate of the ROC curve
analysis. This variance function (\(VF\)) can be expressed using the
\(A\) parameter and \(k\) which we have previously calculated as shown
below:

\[ \begin{aligned}
VF ~ & = ~ 0.0099 ~ \times ~ e ^ {-A ~ \times ~ A/2} ~ \times ~ \Bigg [ ~ (5 ~ \times ~ A ^ 2 ~ + ~ 8) ~ + ~ \frac{(A ^ 2 ~ + ~ 8)}{k} ~ \Bigg ] \\
\\
where: & \\
\\
A ~ & = ~ 1.812114 \\
\\
k ~ & = ~ 4
\end{aligned}\]

Using the values calculated for \(A\) and \(k\) previously, we calculate
\(VF\) as follows:

\[ \begin{aligned}
VF ~ & = ~ 0.0099 ~ \times ~ e ^ {-A ~ \times ~ A/2} ~ \times ~ \Bigg [ ~ (5 ~ \times ~ A ^ 2 ~ + ~ 8) ~ + ~ \frac{(A ^ 2 ~ + ~ 8)}{k} ~ \Bigg ] \\
\\
& = ~ 0.0099 ~ \times ~ 0.193616 ~ \times ~ \Bigg [ ~ (5 ~ \times ~ 3.283757 ~ + ~ 8) ~ + ~ \frac{(3.283757 ~ + ~ 8)}{4} ~ \Bigg ] \\
\\
& = ~ 0.001916798 ~ \times ~ [ ~ 24.41878 ~ + ~ 2.820939 ~] \\
\\
& = ~ 0.001916798 ~ \times 27.23972 \\
\\
& = ~ 0.05221305
\end{aligned} \]

\hypertarget{calculating-number-of-cases-needed-in-the-study-sample}{%
\subsection{Calculating number of cases needed in the study
sample}\label{calculating-number-of-cases-needed-in-the-study-sample}}

Now we can calculate the number of cases (i.e.~those who have the
condition of interest). For the case of ICFI or IYCF, this will be those
children who exhibit or demonstrate appropriate infant and young child
feeding. This can be calculated as follows:

\[ \begin{aligned}
N ~ & = ~ \frac{Z_{\alpha/2} ^ 2 ~ \times ~ VF}{L ^ 2} \\
\\
where: & \\
\\
Z_{\frac{\alpha}{2}} ~ & = ~ 1.96 ~ \text{for a 95\% CI} \\
\\
L ~ & = 0.05 ~ \text{(desired half-width of the CI)} \\
\\
VF ~ & = ~ 0.05221305  \\
\end{aligned} \]

Using the value for \(VF\) calculated previously, we calculate \(N\) as
follows:

\[ \begin{aligned}
N ~ & = ~ \frac{Z_{\alpha/2} ^ 2 ~ \times ~ VF}{L ^ 2} \\
\\
& = ~ \frac{1.96 ^ 2 ~ \times ~ 0.05221305}{0.05^ 2} \\
\\
& = ~ 80.23268 ~ \approx ~ 81
\end{aligned} \]

\hypertarget{total-sample-size}{%
\subsection{Total sample size}\label{total-sample-size}}

The total sample size \(n\) needed for the ROC analysis can then be
calculated as follows:

\[ \begin{aligned}
n ~ & = ~ N ~ \times ~ (1 ~ + ~ k)\\
\\
where: & \\
\\
N ~ & = ~ 81 \\
\\
k ~ & = ~ 4
\end{aligned} \]

Using the \(N\) and \(k\) values calculated earlier, \(n\) is calculated
as follows:

\[ \begin{aligned}
n ~ & = ~ N ~ \times ~ (1 ~ + ~ k)\\
\\
& = ~ 81 ~ \times ~ 5 \\
\\
& = ~ 405
\end{aligned} \]

We will need a sample size of about \textbf{405} children about
\textbf{81} of which will be children who are practising appropriate
infant and young child feeding.

\newpage

\renewcommand\refname{References}
\bibliography{bibliography.bib}


\end{document}
